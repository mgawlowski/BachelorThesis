\chapter*{Wstęp}

Kto nie marzył kiedyś o inteligentnym robocie wyglądającym jak człowiek, poruszającym się jak człowiek, ale przede wszystkim myślącym jak człowiek? Taki robot podczas pierwszego uruchomienia nie miałby żadnej wiedzy czy umiejętności. Jego akcje ograniczałyby się wyłącznie zmysłów. Do zmysłu wzroku -- poprzez kamery, dotyku -- poprzez różne sensory czy słuchu -- poprzez mikrofony. Wszelkie narzędzia do badania świata stałyby się źródłem wiedzy.

Pierwsze oględziny świata opierałyby się zapewne na obserwowaniu, w następnej kolejności -- w momencie zaznajomienia się z otoczeniem robot zapewne zacząłby angażować kolejne zmysły, aż w końcu nauczyłby się wymieniać poglądami z innymi robotami lub ludźmi. Warto się jednak zatrzymać już na samym początku -- etapie obserwowania. Pozornie prosta czynność z perspektywy człowieka. Natomiast z perspektywy robota -- wręcz odwrotnie. Kłopot pojawia się więc już na początku wymarzonego robota.

Główną przeszkodą dla samodzielnego, inteligentnego robota jest mózg. W jaki sposób zasymulować działanie tak skomplikowanego organu? W jaki sposób można by nadać sens słowom i połączyć je z obrazami, które się widzi? Jak rozpoznawać cechy wspólne obiektów, łączyć je w grupy, hierarchie? Takie rozważania nie miałyby końca, a i tak zahaczają na razie o najniższy poziom abstrakcji.

Problem naśladowania pracy mózgu, jego poznawczej natury jest związany tak jak ta praca z pojęciem kognitywistyki, nauki o zmysłach, mózgu i umyśle. Jak do tej pory nie powstało rozwiązanie wyżej nakreślonego problemu, a nawet trudno wierzyć żeby było ono w zasięgu.


\section*{Opis problemu}

Praca dotyczy agenta kognitywnego, którego zadaniem jest obserwacja świata, w którym został umieszczony, a konkretniej obiektów w nim obecnych. Na podstawie zebranych obserwacji i wiedzy z nich zdobytej jest zdolny do odpowiadania na pytania o cechy obiektów, które zaobserwował. 

Jest to zatem podstawowa idea poznawania świata, a zarazem próba naśladowania procesów poznawczych ludzkiego mózgu. Problemem jest złożoność zagadnienia zarówno pod względem teoretycznym, jak i później podczas projektu i implementacji.


\section*{Cel i zakres pracy}

Celem pracy jest projekt i implementacja nowego modułu ekstrakcji cech do aplikacji opracowanej w ramach kursu ZPI (Zespołowe Przedsięwzięcie Inżynierskie), co poprawi wydajność systemu, a także rozszerzy o kilka istotnych funkcji takich jak zapamiętywanie już przeliczonych przekonań na temat obiektów. 


\section*{Treść pracy}

Praca składa się z pięciu rozdziałów, wstępu, zakończenia oraz spisu literatury, rysunków i listingów. 
W rozdziale pierwszym przedstawiono podstawy teoretyczne potrzebne do zrozumienia zagadnienia oraz pracy agenta.
W rozdziale drugim stopniowe wskazano na problemy w istniejącym systemie, zaczynając od praktycznego opisu aplikacji, poprzez przykład pracy, aż do określenia wymagań.
Rozdział trzeci i czwarty przedstawia techniczne elementy aplikacji oraz nowego modułu. Zawierają opis zaproponowanego rozwiązania wraz z przykładami.
Rozdział piąty dotyczy testów systemu. Sprawdzono czy odpowiedzi agenta są takie jakich oczekiwano oraz zbadano wpływ nowego modułu na wydajność agenta.