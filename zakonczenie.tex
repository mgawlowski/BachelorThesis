\chapter*{Zakończenie}

Celem pracy było zaprojektowanie oraz zaimplementowanie nowego modułu ekstrakcji podsumowań danych, będącego częścią kognitywnego agenta zdolnego odpowiadać na zadane przez użytkownika pytania.

Przeanalizowano pracę agenta i zdefiniowano wymagania, które spełniać powinien nowy moduł. Wśród nich było: aktualizowanie holonów z użyciem wyłącznie nowych danych, zapamiętywanie holonów na wypadek wyłączenia lub awarii systemu oraz cykliczne uruchamianie procesów generowania i aktualizowania holonów w trakcie bezczynności.

Wszystkie wymagania zostały zrealizowana w opracowanym module ekstrakcji podsumowań danych. Potwierdziły się również oczekiwania co do poprawy wydajności systemu, co potwierdzone zostało odpowiednimi testami.

Aplikacja wciąż ma wiele możliwości na dalszy rozwój, ale skupiając się na omawianym module ekstrakcji podsumowań danych dobrym kierunkiem byłoby wprowadzenie bardziej zaawansowanego sposobu aktualizacji holonów. Jednym z potencjalnych rozwiązań jest przeprowadzanie analizy dynamiki zmian przekonań. W momencie mocnej zmiany tendencji nowe doświadczenia miałyby większy wpływ na podsumowania niż wcześniejsze. Przy obecnym rozwiązaniu trudno jest zmienić raz mocno ugruntowane przekonania (poparte dużą ilością obserwacji).